\section{Requirements} \label{sec:reqs}

Software producted in LSST Data Management has different stakeholders, that may have different requirements in terms of release.
In this section we try to give an overview of the release requirements for each of them.


\subsection{Requirements from the community } \label{sec:comreqs}

To be able to update to stable bugfix releases in a timely manner
To be able to move to verions with new finctionality in a time;y manner.
To do this in a stable manner (not breaking API changes, schema changes,

\textit{ I would suggest that the science community do not require official releases, but stable distributions.
In this way, it will be much easier to make available new distributions that include expected fixes, 
but do not require formal releses.
To note also, that what the science community is more interested is the availability of the Science Pipelines,
that in reality is not a SW produce but a distribution, that includes many more object.  }

The appendices for the \appref{sec:bib} and \appref{sec:acronyms} are defined in the main file LDM-672.tex.

The main bibliography file is in the lsst-texmf/texmf/bibtex/bib so you can refer 
to documents such as \citeds{LDM-294} (from lsst.bib)  or papers like \cite{2008arXiv0805.2366I} (from refs\_ads.bib). You may make a PR to add new refs to these files.

Acronyms like BOE and BAC will be picked up by generateAcronyms.py which is in lsst-texmf/bin -- that needs to be in the PATH.


\subsection{Processing Release Requirements} \label{sec:procreqs}

NCSA is in charge to host and operate the data processing.

DM developed software released for NCSA processing will depend on the specific processing need.

Release frequencies depends on the processing type:

\begin{itemize}
\item prompt processing has to answer quickly to request for changes, that can happen on a daily bases (official releases)
 or more frequent patch release in case of problems.
\item DRP processing may have a more relaxed schedule, for example on yearly, and can ben rehearsed before running the formal processing.
\end{itemize}

\textit{In order to properly set-up the release process, the SW products to operate need to be identified}

Patch releases need to be provided with a frequency that depends on the type of processing
and on the ugency of the problems to be fixed.


\subsection{Infrastructure Release Requirements} \label{sec:infreqs}

A subset of DM software products will not be used for data processing,
nor will be available to the community for further investigation and contribution to the LSST science.
One example if the software required to implement the LSP.

The releases in this cases are oriented to have a stable infrstructure.
The release candence is n prionciple not tight to processing milestones,
except in case of new functionality to be provided in a planned mmaner.

Pach releases need to be proveded depending on the severity of the problems to be fixed.


\subsection{Other Release Requiremtns} \label{other:reqs}

Other LSST subsystem may be consumers of DM software products.

In order for DM to be able to answer correctly to requests for changes, 
it is important to identify the DM Software Products used operationally by other subsystems.

\newpage
\section{General Considerations} \label{sec:considerations}

The section outlines needs and usage of the LSST science community that drive the definition of the processes 

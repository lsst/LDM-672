\section{Release Management} \label{sec:process}

The DM-CCB maintains the release plan, LDM-564, synchronized with the project milestones.
\footnote{As of June 2019 the release plan needs to be reviewed and the release milstones listed therein need to be made consistent with the scope of the document.
Issue \jira{DM-17001} is tracking this activity.}

Any unscheduled release, major, minor or patch, needs to be requested to the DM-CCB using a RFC Jira issue.
The RFC shall contain:

\begin{itemize}
\item The justification of the release.
\item The date the release is requested to be available.
\item A list of functionalitis or fixes (jira issues) which are requested to be implemented in the release.
\end{itemize}

The DM-CCB  will assess the release request within one week. 
If the release is urgent, DM-CCB will assess it  within 24 hours.
The DM-CCB will approve or reject the proposed release and add a comment to the RFC with the reason of rejection or, in case of approval, with the following information:

\begin{itemize}
\item The release identifier (version number M.N.p).
\item The estimated release date.
\item The list of Jira issues that will be included.
\end{itemize}

In the case an immediate fix is required for a critical operations activity, that can't wait a formal approval of the DM-CCB (24 hours),
a quick decision can be taken by the \gls{DMPM} and a requested fix can be implemented and released immediately, if feasible.
However, an \gls{RFC} has to be filed a posteriori, and the DM-CCB is required to assess it.


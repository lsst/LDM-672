\section{Process} \label{sec:process}

\subsection{Release Process Definition}

The release process will be as follows:

\begin{itemize}
\item Major and minor releases will be planned and approved by the DM-CCB.
\begin{itemize}
  \item  DM-CCB will provide  a release plan identifying  milestones requiring software releases.
  \item The DM-CCB will monitor and approve or reject the content of each release. DM-CCB will ensure that  all RFCs that impact  release content are properly escalated.
\end{itemize}
\item Any extra releases, major, minor or patch, has to be requested by the release end users to the DM-CCB via \gls{RFC}. The \gls{RFC} shall contain:
\begin{itemize}
  \item the reason for an extra release
  \item the list of features or fixes which are requested to be implemented in the release
  \item the date the release is requested to be available and specify the urgency.
\end{itemize}
\item The DM-CCB will assess the release request involving the relevant T/CAMs and technical contributor affected.
Urgent release request will be approved in 24 hours, non urgent release requests will be approved in one week's time.
\item The release request may be approved or rejected.
\item In case of approval, the following information will be added in a comment to the RFC:
\begin{itemize}
  \item version identification of the release
  \item date the release will be available (an estimation, that may change)
  \item confirm content of the release
\end{itemize}
\item the DM-CCB will monitor the progress of the release's activity and communicate changes in the release date
\end{itemize}


The DM-CCB meets every week.

Technical aspects and roles (who does what) are addressed int following documents:

\begin{itemize}
\item {\bf regular development}: in the \href{https://developer.lsst.io/}{developer guide}
\item {\bf fix porting}: in the \href{https://developer.lsst.io/}{developer guide}
\item {\bf release note documentation}: in the \href{https://developer.lsst.io/}{developer guide} or a different document
\item {\bf release engineering}: in the DMTN-106 (still draft). \href{https://sqr-016.lsst.io/}{SQR-016} is used for the science pipelines release.
\end{itemize}


\subsection{Urgent Releases}

In the case a very urgent release is required, that can't wait a formal approval of the DM-CCB (24 hours),
a quick decision can be taken by the \gls{DMPM} and a requested fix can be implemented and released immediately, if feasible.

However, an \gls{RFC} has to be filed a posteriori, and the DM-CCB is required to assess it.


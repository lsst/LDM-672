\section{Release High Level Process} \label{sec:process}

The DM-CCB maintains the release plan, \citeds{LDM-564}, synchronized with the project milestones.
\footnote{As of June 2019 the release plan needs to be reviewed and the release milstones listed therein need to be made consistent with the scope of the document.
Issue \jira{DM-17001} is tracking this activity.}

Any unscheduled release --- major, minor or patch, and including a request to backport fixes to an earlier release --- must be requested of the DM-CCB using a Jira issue of type ``RFC''.
The issue shall describe:

\begin{itemize}
\item The justification for the release.
\item The date by which the release should be available.
\item A list of specific functionality and/or bug fixes which should be contained in the release, specified in terms of Jira tickets.
\end{itemize}

The DM-CCB will assess the release request within one week.
If the release is urgent, DM-CCB may assess it within 24 hours.
The DM-CCB will approve or reject the proposed release and add a comment to the RFC with the reason of rejection or, in case of approval, with the following information:

\begin{itemize}
\item The release identifier (version number M.N.p).
\item The estimated release date.
\item The list of Jira issues that will be included.
\end{itemize}

In the case an immediate fix is required for a critical operations activity, a quick decision can be taken by the DM Project Manager and a requested fix can be implemented and released as rapidly as possible.
However, an \gls{RFC} has to be filed a posteriori, and the DM-CCB is required to review it.

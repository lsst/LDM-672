\section{Process} \label{sec:process}

\subsection{Release Process Definition}

The release process will be as follows:

\begin{itemize}
\item Major and minor releases will be planned and approved by the DM-CCB. 
\begin{itemize}
  \item It is recommended that the DM-CCB provide on a yearly base a release plan with a tentative schedule and release content.
  \item The DM-CCB will monitor and approve or reject the content of each release. This will be done ensuring that all RFC that impacts a release content are approved by the DM-CCB.
\end{itemize}
\item All releases will be clearly identified in Jira using the Jira functionality. 
If this is not available, a {\it Release Issue} will be created and all issues that are blocking the release will be related as blocker to it. 
\item Any extra releases, major, minor or patch, has to be requested by the release end users to the DM-CCB via RFC. The RFC shall contain:
\begin{itemize}
  \item reason an extra release
  \item list of features or fixes which are quested to be implemented in the release
  \item date the release is requested to be available and specify the urgency.
\end{itemize}
\item The DM-CCB will assess the release request involving the relevant T/CAMs and technical contributor affected. 
Urgent release request will be approved in 24 hours, non urgent release requests will be approved in one week's time.
\item The release request may be approved or rejected.
\item In case of approval, the following information will be added to a comment in the RFC:
\begin{itemize}
  \item version identification of the release
  \item date the release will be available (an estimation, that may change)
  \item confirm content of the release
\end{itemize}
\item the DM-CCB will monitor the progress of the release's activity and communicate changes in the release date
\end{itemize}


The DM-CCB will meet every week.

Technical aspects and roles (who does what) are addressed int following documents:

\begin{itemize}
\item {\bf regular development}: in the \href{https://developer.lsst.io/}{developer guide}
\item {\bf fix porting}: in the \href{https://developer.lsst.io/}{developer guide}
\item {\bf release documentation}: in the \href{https://developer.lsst.io/}{developer guide} or a different document
\item {\bf release engineering}: in the DMTN-106 (still draft). \href{https://sqr-016.lsst.io/}{SQR-016} is used for the science pipelines release.
\end{itemize}


\subsection{Release Timing}

Preparing a release is a compless process.
The time needed to complete it depends on many factors:

\begin{itemize}
\item assessment and development time
\item build, continuous integration (CI) and validation time
\item communication delays
\item CCB process time
\end{itemize}

This list is not complete. Other unespected factors may enter in the loop.

Defining a clear process, as done in this document, will help to reduce to the minimum the communication delays and CCB process time.

Technical time, needed to assess and develop the solution, may be very difficult to reduce.

Build, CI and validation time are depending from the infrastructure, architecture and tooling used. 


\subsection{Urgent Releases}

In the case a very urgent release is required, that can't wait a formal approval of the DM-CCB (24 hours), 
a quick decision can be taken by the DMPM and a requested fix can be implemented and released immediatelly, if feaseble.

However, a RFC need to be filed, a posteriori, and the DM-CCB is required to assess it.


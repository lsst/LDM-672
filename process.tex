\section{Process} \label{sec:process}

The release process will be as follows:

\begin{itemize}
\item Major and minor releases will be planned and aproved by the DM-CCB well in advance. 
\begin{itemize}
  \item It is recommended that the DM-CCB provide on a yearly base a release plan with tentative release dates and contents.
\end{itemize}
\item all releases sould be clearly identified in Jira using the Jira functionality. If this is not available, a {\it Release Issue} ca be created (by the CRM) and the issues that are blocking the release, shall be related as blocker to the Release Issue. 
\item Any extra releses, major, minor or patch, need to be requested by the release end users to the DM-CCB via RFC. The RFC shall contain:
\begin{itemize}
  \item the reason an extra release
  \item the the list of features or fixes that need to be implemented in the release
  \item when the release should be available, and specify if it is urgent or not.
\end{itemize}
\item The DM-CCB will assess the release request, involving the relevant T/CAMs and tecnical contributor affected. 
The DM-CCB shall provide mechanisms to approve a relese depending on the urgency. 
Urgent releases coul be aproved in 24 hours, non urgent releases can be approved in one week time.
\item The release request may be approved or rejectd.
\item In case of approval, the following informatino will be provided back:
\begin{itemize}
  \item The version identification of the release
  \item an estimatimation when the release will be available
  \item confirm the content of the release
\end{itemize}
\item the DM-CCB will monitor that the release will be ready on time and comunicate changes in the release date
\end{itemize}


The DM-CCB is suppose to meet every week.

Thecnical aspects and roles (who does what) shall be addressed as follows:

\begin{itemize}
\item {\bf regular development}: in the \href{https://developer.lsst.io/}{developer guide}
\item {\bf fix porting}: in the \href{https://developer.lsst.io/}{developer guide}
\item {\bf documentation}: in the \href{https://developer.lsst.io/}{develiper giuids} or a different document
\item {\bf release engineering}: in a general release document (to be writte). At the time been \href{https://sqr-016.lsst.io/}{SQR-016} will be used for the science pipelines release.
\end{itemize}



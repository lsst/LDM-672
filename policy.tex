\section{Release Policy} \label{sec:policy}

The following policy is provided taking into account the requirements described in the previous section \ref{sec:reqdef}.


\subsection{Versioning Policy} \label{sec:versinopolicy}

The DM release versioning shall follow Semantic Versioning\footnote{\url{https://semver.org/}} as described in \citeds{DMTN-106}, section 3.3.

This policy answers to \ref{sec:stable} requirement.
However, API stability is a development aspect that needs to be addressed tin the \citeds{DevGuideL}.


\subsection{Release Schedule Policy} \label{sec:schedulepolicy}

Major and minor releases should be scheduled accordingly to the requirements of each stakeholder.

A release plan shall be provided according to \citeds{LDM-294}, that specify for each software product:

\begin{itemize}
\item when a release is expected
\item specify the milestone or RFC relevant that requested the release
\item if it is a time-based release: specify the timing when the release shall be done (for example, every 6 months)
\item if it is a functionality based release: specify which functionalities shall be included in the release
\item specify the list of know stakeholders
\end{itemize}

This policy answers to \ref{sec:milestone} requirement.


\subsection{Patch Releases and Backportingn Policy} \label{sec:patchpolicy}

Patch releases shall be scheduled, tested and approved by the DMCCB as much as possible.

However, the DMPM can authorize urgent changes to the operational systems in case of need. 
This is already stated in \citeds{LDM-294} section 3.6.

The backporting capability is a development aspect that needs to be addressed in the \citeds{DevGuideL}.

This policy answers to \ref{sec:backport} requirement.


\subsection{Release Testing Policy} \label{sec:testpolicy}

All releases should be fully tested and test activities documented using Jira ATM.

The scope of the test is to ensure that:

\begin{itemize}
\item the planned functionalities are provided and working
\item there is no regression (reintroduction of previously fixed issues)
\end{itemize}

LSST System Engineering provides in Jira an integrated infrastructure (ATM) that makes easy to document and execute test activities.

This policy answers to \ref{sec:test} requirement.

This policy is not mandatory.
In some cases unit tests may be considered sufficient or, it may not be possible to fully test a software release in isolation.


\subsection{Release Note Policy} \label{sec:notepolicy}

Each release shall be documented by a Release Note.

It shall always be possible to collect automatically a minimum set of information to include in the release note and therefore properly identify the release.

\citeds{DMTN-106} subsection 3.2 (see \ref{sec:defs}) provides a definition of software release note that can be used as is or tailed depending on the needs.

This policy answers to \ref{sec:reqdoc} requirement.


\subsection{Third Party Software Policy} \label{sec:thirdpolicy}

In shall be possible to include in a software product or distribution a software package provided by an external collaborator.

The use of all third party packages shall be approved by the DMCCB using the RFC mechanism.

Following ways to include external software can be used:

\begin{itemize}
\item include the package in the environment, if used in the DM code
    \begin{itemize}
    \item this imply that the 3rd party package is available in conda, if not DM should provide a recipe for that.
    \item technicalities to be addressed in the environment management procedures
    \item the software is provided \textit{as is} and DM is not responsible for it
    \end{itemize}
\item include the package in a distribution, as an additional EUPS package 
    \begin{itemize}
    \item technicalities to be addressed in the \citeds{DevGuideL}
    \item the software is provided \textit{as is} and DM is not responsible for it
    \end{itemize}
\item include the package in a software product, in case it requires specific LSST customization
    \begin{itemize}
    \item technicalities to be addressed in the \citeds{DevGuideL}
    \item DM is taking responsibility for the customized package.
    \end{itemize}
\end{itemize}

An internal person shall be identified as point of contact and reference for the external package.

This policy answers to \ref{sec:thirdsw} requirement.


\newpage
\section{Non Compliance} \label{sec:noncompliance}

The policies described in the previous section \ref{sec:policy} are mandatory, except where specified as non-mandatory.

Each software product responsible (T/CAM) shall specify which policy is not applicable for the specific software product and why.

Also, each stakeholder shall ensure that the level of compliance provided by the required software product is acceptable for them.

The DM-CCB shall ensure that compliance information is available and agreed between T/CAMs and stakeholders.

It has to be taken in account that, each time a policy is not considered applicable to a software product, this will reduce the effectiveness of the process and increase the risks of problems to be experienced by the stakeholders.

\documentclass[DM,lsstdraft,toc]{lsstdoc}

% lsstdoc documentation: https://lsst-texmf.lsst.io/lsstdoc.html

% Package imports go here.

% Local commands go here.

% To add a short-form title:
% \title[Short title]{Title}
\title{LSST Software Release Management}

% Optional subtitle
% \setDocSubtitle{A subtitle}

\author{%
Leanne P. Guy, Gabriele Comoretto
}

\setDocRef{LDM-672}

\date{\today}

% Optional: name of the document's curator
\setDocCurator{Leanne Guy }

\setDocAbstract{%
In this document we propose a process for LSST software releases that will enable early adopters in the LSST science community to work with and contribute to the LSST software. 
}

% Change history defined here.
% Order: oldest first.
% Fields: VERSION, DATE, DESCRIPTION, OWNER NAME.
% See LPM-51 for version number policy.
\setDocChangeRecord{%
  \addtohist{1}{YYYY-MM-DD}{Unreleased.}{Leanne P. Guy}
}

\begin{document}

% Create the title page.
% Table of contents is added automatically with the "toc" class option.
\maketitle



% ADD CONTENT HERE ... a file per section can be good for editing
\input{body}

\appendix
% Include all the relevant bib files.
% https://lsst-texmf.lsst.io/lsstdoc.html#bibliographies
\section{References} \label{sec:bib}
\bibliography{lsst,lsst-dm,refs_ads,refs,books}

%Make sure lsst-texmf/bin/generateAcronyms.py is in your path
\section{Acronyms used in this document}\label{sec:acronyms}
\addtocounter{table}{-1}
\begin{longtable}{|l|p{0.8\textwidth}|}\hline
\textbf{Acronym} & \textbf{Description}  \\\hline

API & Application Programming Interface \\\hline
BAC & Budget At Completion \\\hline
BOE & Basis Of Estimate \\\hline
CCB & Change Control Board \\\hline
CI & Continuous Integration \\\hline
DM & Data Management \\\hline
DMPM & Data Management Project Manager \\\hline
DMTN & \gls{DM} Technical Note \\\hline
DRP & Data Release Production \\\hline
LDM & \gls{LSST} Data Management (document handle) \\\hline
LSP & \gls{LSST} Science Platform \\\hline
LSST & Large Synoptic Survey Telescope \\\hline
NCSA & National Center for Supercomputing Applications \\\hline
OSX & Macintosh Operating System \\\hline
PR & Pull Request \\\hline
RFC & Request For Comment \\\hline
SQR & SQuARE document handle \\\hline
SW & Software (also denoted S/W) \\\hline
\end{longtable}

\end{document}

\documentclass[DM,lsstdraft,toc]{lsstdoc}
\usepackage{bbding}

% lsstdoc documentation: https://lsst-texmf.lsst.io/lsstdoc.html

% Package imports go here.

\input meta.tex

\makeglossaries
\input{aglossary.tex}

% Local commands go here.

% To add a short-form title:
% \title[Short title]{Title}
\title{LSST Software Release Management}

\author{%
G. Comoretto, L.~P.~Guy, W. O'Mullane, K.-T. Lim, M. Butler, M. Gelman, J.D.~Swinbank
}

\setDocRef{LDM-672}

\date{\today}

% Optional: name of the document's curator
\setDocCurator{Gabriele Comoretto}
\setDocUpstreamLocation{\url{https://github.com/lsst/ldm-672}}
\setDocUpstreamVersion{\vcsrevision}

\setDocAbstract{%
This document outlines the policies and high level management approach for \gls{LSST} \gls{DM} software product releases.
}


% Change history defined here.
% Order: oldest first.
% Fields: VERSION, DATE, DESCRIPTION, OWNER NAME.
% See LPM-51 for version number policy.`
\setDocChangeRecord{%
  \addtohist{}{2018-12-03}{First draft}{L.~Guy, G.~Comoretto}
  \addtohist{1.0}{2019-07-XX}{First issue, approved on RFC-619}{G.~Comoretto}
}

\begin{document}

% Create the title page.
% Table of contents is added automatically with the "toc" class option.
\maketitle

% ADD CONTENT HERE ... a file per section can be good for editing
\section{Introduction} \label{sec:intro}

The current situation is  that
\begin{itemize}
\item Stable releases every 6 months
\item Weekly builds
\item Nightly builds
\end{itemize}

The current release process requires few weeks to make an official release available.

For this reasons weekly builds are largerly used, since they provide the most up to date and stable software.




\newpage
\section{Stakeholders Requirements} \label{sec:reqs}

This section gives an overview of requirements on \gls{DM} software product releases from the different stakeholders.

It is important to identify the expected release requirements and the corresponding policy during the construction phase, in order to have it consolidated when operations start.

In operations, some of the \gls{LSST} subsystems may no longer exist, for example, \gls{DM} will not exist per se but \gls{DM} software products will still exist under other managerial structure.
The policy defined here will still be applicable since many of the stakeholders will still expect software releases to be managed in the same manner as during construction.


\subsection{Release Requirements for the \gls{LSST} Science Community} \label{sec:comreqs}

In preparation for working with the \gls{LSST} data products and software during operations, several \gls{LSST} science collaborations have begun using the \gls{DM} software to run data challenges using precursor data or simulations, and to do performance studies. These activities effectively increase the number of beta-testers of \gls{DM} software products, providing valuable feedback to \gls{DM} on the state of the system. 

In order to work effectively with the \gls{DM} software while it is still under development, the science community require: 
\begin{itemize}
\item stable public APIs and schemas in order to build software for User-Generated analyses, 
\item patches and bug fixes back-ported to the current stable version of the software,
\item include software provided by external contributors in a software release or distribution (due to the collaborative nature of the project).
\end{itemize}


\subsection{Release Requirements for Data Processing in Operations} \label{sec:procreqs}

The \gls{LDF} will be responsible for generating the necessary data products during commissioning and operations. 
The \gls{LDF} requires officially released software to be used in production for the various operational activities.
Software releases will be run in production at National \gls{Center} for Supercomputing Applications (\gls{NCSA}), CC-IN2P3 and in Chile, and possibly at independent Data Access Centers (iDACs).

Release frequencies will depend on the processing type:
\begin{itemize}
\item Prompt Processing requires releases to be available very quick, that could happen on a daily bases, especially in the early days (official releases)
 or a  patch release  for a specific problem may be required during the night (this would require sign off from the \gls{AD} for Science \gls{Operations}).
\item \gls{DRP} processing  must be stable for long periods, currently processing is foreseen to take 9 months.  Before such a long processing run the release must be very well tested and any updates extremely well controlled.
\item Image acquisition and header service, which form part of the image acquisition on the mountain will also need strict change control. Releases for this could be on monthly or even lounge timescales - however if there is a problem a patch will be needed immediately.
\end{itemize}

Patch releases need to be provided with a frequency that depends on the type of processing
and on the urgency of the problems to be fixed.


\subsection{Release Requirements for \gls{LSST} Subsystems in \gls{Operations}.} \label{sec:otherreqs}

Other \gls{LSST} subsystem may be consumers of \gls{DM} software products, for example the Telescope \& Site subsystem software makes use of the \gls{DM} software products. 
In order for \gls{DM} to be able to respond correctly to the needs of other \gls{LSST} subsystems, it is important to first identify which \gls{DM} software products are used, and how they are used.


\subsection{Release Requirements for Infrastructural Software} \label{sec:infreqs}

A significant subset of \gls{DM} software products are used to provide services to \gls{LSST} science users and staff but are not directly used to generate \gls{LSST} science data products. An example of this is the software that implements the \gls{LSST} \gls{Science Platform} (\gls{LSP}).

Releases of this type of software are typically on their own cadence and need to be adequately tested before deployment to ensure a stable infrastructure. 
The releases may be tied to processing milestones if the services or features thereof are required for the processing (e.g. functionality of the workflow service may be required for \gls{Data Release} processing and features in the \gls{LSP} may be needed for \gls{QA} of data products).

Patch releases may need to be provided depending on the urgency and severity of the problems to be fixed.


\subsection{Release Requirements for non-Operational Project Activities} \label{sec:nonopsreqs}

This includes activities done in preparation for operations, such as commissioning, 
large scale integration/validation test campaigns, etc. These activities should use use,  as much as possible,  officially released software.

In some cases however, it is necessary to use non-released software, such as release candidates or stable builds.
In all cases, the software used must be clearly identified (Github revision at least), and the distribution/deployment strictly controlled.


\newpage
\section{Requirements Consolidation} \label{sec:reqdef}

The following list of unique requirements is derived from the above section \ref{sec:reqs}.

The main purpose of this section is to clearly identify and expand those requirements.
In addition, a few general requirements are given, that are not specific to any stakeholder.

A summary overview of the requirements per stakeholder is given at the end.


\subsection{Release General Requirements} \label{sec:genreq}

The following requirements are needed in order to properly implement the release process.


\subsubsection{Software Products Identification Requirement} \label{sec:swid}

All \gls{SW} products shall be clearly and unequivocally identifiable in the source repository (GitHub) and documented.

\citeds{DMTN-106} subsection 2.2 (see \ref{sec:defs}) provides a software product definition that can be used as a starting point to identify the DM software products.
The DM product tree provided together with \citeds{LDM-294}, is available at the following link \url{https://ldm-294.lsst.io/ProductTreeLand.pdf}.

This requirement needs to be fulfilled in order to ensure the applicability of the release policy and process.
If the software products are not properly identified, it will not be possible to do releases.


\subsubsection{Software Release Documentation Requirement} \label{sec:reqdoc}

All software releases shall be properly documented with a software release note.


\subsubsection{Software Release Test Requirement} \label{sec:test}
 
A software release should be fully tested before making it available for use.
The test should be documented in a test report.


\subsection{Releases Schedule Requirement} \label{sec:milestone}

Releases on a software product shall be scheduled in advance.

Two types of release schedule can be identified:

\begin{itemize}
\item Functional Based Release Schedule: a release shall provide a requested functionality.
\item Time-Based Release Schedule: a release shall be provided on a specific date or cadence.
\end{itemize}

In both cases, releases can be tight to project milestones.
Additional releases can be requested to the DM-CCB using the RFC mechanism.


\subsection{Patch Release Requirement} \label{sec:backport}

It shall be possible to backport a fix on a stable release and provide a patch release including only the backported fix.


\subsection{Third Part Software Inclusion Requirement} \label{sec:thirdsw}

It shall be possible to include in a software product release or distribution release, a software package provided by a third party contributor.

Reasons to include third-party code in DM software or distributions are:
\begin{enumerate}
\item There are packages to  use in DM code\footnote{E.g. starlink\_ast}.  \label{item:depend}
\item There are packages to use when working with the deployed software\footnote{Pandas  might be such an example}, - they are not needed it to build or run DM code but they may provide extra functionalities for interaction with the data.\label{item:want}
\item Collaborator's packages for data processing or postprocessing\footnote{that depend on DM code or provide algorithm DM may want to use, like for example ngmix}. This is what most people on LSST are thinking about under the topic of third party software. \label{item:colab}
\end{enumerate}


\subsection{Stable public APIs and schemas} \label{sec:stable}

Public APIs and schemas shall be stable and follow a well-defined deprecation mechanism in order to give time to the stakeholder to adapt to the new API.


\subsection{Requirements Summary Overview} \label{sec:overview}

The following table gives an overview of the release requirements applicable for each stakeholder.

\setlength\LTleft{-0.35in}
\setlength\LTright{-0.5in}
\begin{longtable}{p{2.5cm}p{1.5cm}p{1.5cm}p{1.5cm}p{1.5cm}p{1.5cm}p{1.5cm}p{1.7cm}p{1.5cm}}\hline
& 
SW Ident.            & Release Doc.                  & Release Test & Funct. Based           & Time Based & Patch & 3rd Party SW  & Stable API  \\ \hline
Science Community &
YES                     & YES                                &                       &                                 &  YES            & YES.   & YES               &  YES.          \\ \hline
DM Operations &
YES                     & YES                                & YES               & YES                         &                     & YES.   & YES               &  YES(?)      \\ \hline
Other Operations &
YES                     & YES                                & YES               & YES                         &                     & YES.   &                       &  YES(?)      \\ \hline
Infrastructure &
YES                     & YES                                & YES               &                                 &  YES            & YES.   &                       &  YES           \\ \hline
Non Operations &
YES                     & YES(?)                            & YES(?)          & YES                         &  YES            & YES.   & YES               &  YES          \\ \hline
\hline
\end{longtable}
\setlength\LTleft{0in}
\setlength\LTright{0in}




\newpage
\section{Release Policy} \label{sec:policy}

The following policies are derived from the consolidated requirements described in the previous section \ref{sec:reqdef}.


\subsection{Versioning Policy} \label{sec:versinopolicy}

The DM release versioning shall follow Semantic Versioning\footnote{\url{https://semver.org/}} as described in \citeds{DMTN-106}, section 3.3.

This policy partially addresses the requirement for stable APIs (\S\ref{sec:stable}).

% Removed this comment, because it is not release policy.
%In general, API stability is a development aspect that needs to be addressed in the \citep{DevGuide}.


\subsection{Release Schedule Policy} \label{sec:schedulepolicy}

Major and minor releases should be scheduled accordingly to the requirements of each stakeholder.

Each release shall be made following a release plan, which provides the following information:

\begin{itemize}
\item when the release is expected;
\item the specify the corresponding milestone or RFC;
\item if it is a time-based release: specify the release cadence (for example, every 6 months);
\item if it is a functionality based release: specify which features shall be included in the release.
\end{itemize}

This policy addresses requirement \$\ref{sec:milestone}.


\subsection{Patch Releases and Backporting Policy} \label{sec:patchpolicy}

Patch releases shall be scheduled, tested and approved by the DMCCB wherever possible.
However, the DM Project Manager can authorize urgent releases if required.

The mechanisms by which fixes are back-ported are a matter of development practice, rather than policy; as such, they are addressed in the Developer Guide \citep{DevGuide}.

This policy addresses requirement \S\ref{sec:backport} .


\subsection{Release Testing Policy} \label{sec:testpolicy}

All releases must be fully tested and the results recorded according to standard DM practice\footnote{At time of writing, this means that test activities are managed using the Adaptavist Test Management framework for Jira.}.

The scope of the test is to ensure that:

\begin{itemize}
\item All planned functionality is provided and working;
\item There is no regression in functionality or performance relative to previous releases.
\end{itemize}

This policy addresses requirement \S\ref{sec:test} .

Note that, in some cases, and at the discretion of the DM-CCB, it may be appropriate to relax this policy and release based solely on successful execution of unit and/or integration tests.


\subsection{Release Note Policy} \label{sec:notepolicy}

Each release shall be documented by an accompanying set of release notes.

\citeds{DMTN-106} \S3.2 provides a definition of software release note that can be used as is or tailed depending on the needs (see also \S\ref{sec:defs} of this document).

This policy addresses requirement \S\ref{sec:reqdoc}.


\subsection{Third-Party Software Policy} \label{sec:thirdpolicy}

In shall be possible to include in a software product or distribution a software package provided by an external collaborator.
The use of all third-party packages shall be approved by the DMCCB using the RFC mechanism.
External software may be included in the following ways:

\begin{itemize}
\item The package may be included in the standard environment used to execute DM code\footnote{At time of writing, this environment is based on Conda.}
    \begin{itemize}
    \item this implies that the third party software is already packaged for that environment, or, alternatively, that DM must itself provide sucha a package.
    \item the software will be provided \textit{as is} and DM is not responsible for it
    \end{itemize}
\item include the software as part of a DM software distribution, as an additional package \footnote{At time of wrting, this would mean packaging for EUPS.}
    \begin{itemize}
    \item the software is provided \textit{as is} and DM is not responsible for it
    \end{itemize}
\item Create a new DM-provided software product to contain the third-party software
    \begin{itemize}
    \item This is appropriate if DM-specific patches or customization is required.
    \item DM takes responsibility for the customized package.
    \end{itemize}
\end{itemize}

In all cases, a member of the DM team shall be identified as point of contact and reference for the external package.

This policy addresses requirement \S\ref{sec:thirdsw}.


\subsection{Software License Policy} \label{sec:licensepolicy}

All DM-produced software must carry an Open Source Initiative (OSI)-approved license\footnote{\url{https://opensource.org/licenses}}.
Specific requirements for particular aspects of the DM codebase are addressed in the Developer Guide \citep{DevGuide}.


\newpage
\section{Applicability of policies } \label{sec:applicability}

The policies described in \S\ref{sec:policy} of this document are mandatory for all DM software products, as defined in \citeds{LDM-294}, except where otherwise specified.


\subsection{Non-Compliance} \label{sec:noncompliance}

Individual T/CAMs or Product Owners may request that the DM-CCB issue exemptions from specific aspects of policy for particular software products.
The DM-CCB is responsible for ensuring that the resulting system still meets the needs of applicable stakeholders, and for ensuring that the overall integrity of the DM system is maintained.


\newpage
\section{Process} \label{sec:process}

\subsection{Release Process Definition}

The release process will be as follows:

\begin{itemize}
\item Major and minor releases will be planned and approved by the DM-CCB. 
\begin{itemize}
  \item It is recommended that the DM-CCB provide on a yearly base a release plan with a tentative schedule and release content.
  \item The DM-CCB will monitor and approve or reject the content of each release. This will be done ensuring that all RFC that impacts a release content are approved by the DM-CCB.
\end{itemize}
\item All releases will be clearly identified in Jira using the Jira functionality. 
If this is not available, a {\it Release Issue} will be created and all issues that are blocking the release will be related as blocker to it. 
\item Any extra releases, major, minor or patch, has to be requested by the release end users to the DM-CCB via RFC. The RFC shall contain:
\begin{itemize}
  \item reason an extra release
  \item list of features or fixes which are quested to be implemented in the release
  \item date the release is requested to be available and specify the urgency.
\end{itemize}
\item The DM-CCB will assess the release request involving the relevant T/CAMs and technical contributor affected. 
Urgent release request will be approved in 24 hours, non urgent release requests will be approved in one week's time.
\item The release request may be approved or rejected.
\item In case of approval, the following information will be added to a comment in the RFC:
\begin{itemize}
  \item version identification of the release
  \item date the release will be available (an estimation, that may change)
  \item confirm content of the release
\end{itemize}
\item the DM-CCB will monitor the progress of the release's activity and communicate changes in the release date
\end{itemize}


The DM-CCB will meet every week.

Technical aspects and roles (who does what) are addressed int following documents:

\begin{itemize}
\item {\bf regular development}: in the \href{https://developer.lsst.io/}{developer guide}
\item {\bf fix porting}: in the \href{https://developer.lsst.io/}{developer guide}
\item {\bf release documentation}: in the \href{https://developer.lsst.io/}{developer guide} or a different document
\item {\bf release engineering}: in the {\it Release procedure} technical note (to be written). At the time been \href{https://sqr-016.lsst.io/}{SQR-016} will be used for the science pipelines release.
\end{itemize}

\subsection{Release Timing}

Preparing a release is a compless process.
The time needed to complete it depends on many factors:

\begin{itemize}
\item assessment and development time
\item build, continuous integration (CI) and validation time
\item communication delays
\item CCB process time
\end{itemize}

This list is not complete. Other unespected factors may enter in the loop.

Defining a clear process, as done in this document, will help to reduce to the minimum the communication delays and CCB process time.

Technical time, needed to assess and develop the solution, may be very difficult to reduce.

Build, CI and validation time are depending from the infrastructure, architecture and tooling used. 

\subsection{Urgent Releases}

In the case a very urgent release is required, that can't wait a formal approval of the DM-CCB (24 hours), 
a quick decision can be taken by the DMPM and a requested fix can be implemented and released immediatelly, if feaseble.

However, a RFC need to be filed, a posteriori, and the DM-CCB is required to assess it.



\newpage
\appendix

% Include all the relevant bib files.
% https://lsst-texmf.lsst.io/lsstdoc.html#bibliographies
\section{References} \label{sec:bib}
\bibliography{lsst,lsst-dm,refs_ads,refs,books,local}

%Make sure lsst-texmf/bin/generateAcronyms.py is in your path
\printglossaries

\end{document}

\documentclass[DM,lsstdraft,toc]{lsstdoc}
\usepackage{bbding}

% lsstdoc documentation: https://lsst-texmf.lsst.io/lsstdoc.html

% Package imports go here.

\input meta.tex

\makeglossaries
\input{aglossary.tex}

% Local commands go here.

% To add a short-form title:
% \title[Short title]{Title}
\title{LSST Software Release Management}

\author{%
G. Comoretto, L.~P.~Guy, W. O'Mullane, K.-T. Lim, M. Butler, M. Gelman, J.D.~Swinbank
}

\setDocRef{LDM-672}

\date{\today}

% Optional: name of the document's curator
\setDocCurator{Gabriele Comoretto}
\setDocUpstreamLocation{\url{https://github.com/lsst/ldm-672}}
\setDocUpstreamVersion{\vcsrevision}

\setDocAbstract{%
This document outlines the policies and high level management approach for \gls{LSST} \gls{DM} software product releases.
}


% Change history defined here.
% Order: oldest first.
% Fields: VERSION, DATE, DESCRIPTION, OWNER NAME.
% See LPM-51 for version number policy.`
\setDocChangeRecord{%
  \addtohist{}{2018-12-03}{First draft}{L.~Guy, G.~Comoretto}
  \addtohist{1.0}{2019-07-25}{First issue, approved on RFC-619}{G.~Comoretto}
}

\begin{document}

% Create the title page.
% Table of contents is added automatically with the "toc" class option.
\maketitle

% ADD CONTENT HERE ... a file per section can be good for editing
\section{Introduction} \label{sec:intro}

This document prests the release manafgement approach for \gls{LSST} \gls{DM} software products.

First of all, in section \ref{sec:reqs}, the release requirements for the varius  stakeholders are identified.
The release requirements are then consolideated in section \ref{sec:reqdef}.
These requirements are not formal project requirements, as given for example in the \gls{DMS} requirements specification \citeds{LSE-61}, but are nevertheless important to ensure the project's goals.

Based on the consolidated reqirements, a set of policies are derived in section \ref{sec:policy} and guidelines on their applicability is provided in section \ref{sec:noncompliance}.

Finally section \ref{sec:process} goves a high level overview of the release process.


\subsection{Releases Status}\label{sec:sci}

Currently, only the \gls{Science Pipelines} product is released. 
Builds and releases are made on the following time-based cadence:

\begin{itemize}
\item Nightly builds
\item Weekly builds
\item Official releases every 6 months
\end{itemize}

The time needed to consolidate an official release from a weekly build is considerable.
Usually 2 or 3 weeks are sufficient but in some cases it may take more than a month. 
Consequently, by the time a release becomes available to the users, it is already old.
For this reason, users generally prefer to work with weekly builds that are sufficiently stable and include all new functionalities completed in the last week.

The \gls{Science Pipelines} release checklist is documented in \citeds{SQR-016}.
The technical note \citeds{DMTN-106} generalize the process and summarize the technical problems that need to be solve in order that this procedur can be applied to other software products.


\subsection{Policy Applicability} \label{sec:applicability}

This policy shall be applicable to all \gls{DM} software products.

The \gls{DM} software products are defined in the LDM-294. 
The approved product tree is available online at\url{https://ldm-294.lsst.io/ProductTreeLand.pdf}

The policy may also be applicable to other software products development by other \gls{LSST} subsystems.

In case 


\subsection{Definitions} \label{sec:defs}

The relevant definition to be considered when working on release policy and process are given in \citeds{DMTN-106}, section 2.


\newpage
\section{Stakeholders Requirements} \label{sec:reqs}

This section address the questions:

\textbf{Why are DM software releases needed? Who is requesting them ?}

The classic answer to the second questions is that  stakeholders request releases for various reasons.

The following subsections summarize the release requirements on \gls{DM} software products from the different stakeholders' points of view.

It is important to identify these requirements and the corresponding policies during the construction phase, in order to have them consolidated when operations start.

In operations, some of the \gls{LSST} subsystems may no longer exist, for example, \gls{DM} will not exist per se but \gls{DM} software products will still exist under another managerial structure.
The requirements and policies defined here will still be applicable since many of the stakeholders will still expect software releases to be managed following the process consolidated during construction.


\subsection{Release Requirements for the \gls{LSST} Project} \label{sec:lsstreqs}

The main reason for DM to do software releases is to fulfill the project outcome: provide the Data Products.

In order to do that, software releases shall:

\begin{itemize}
\item be available on a time base cadence (every 6 months) during the construction phase,
\item be available when required by a project milestone, in order to be used for the proper validation (commissioning) and operations activities,
\item implement the  functionality needed for the milestone or phase of the project,
\item include third-party software from the Science Community when this is beneficial to the project outcome,
\item carry an appropriate source code license to permit contributions from the Science Community.
\end{itemize}


\subsection{Release Requirements for the \gls{LSST} Science Community} \label{sec:comreqs}

In preparation for working with the \gls{LSST} data products and software during operations, several \gls{LSST} science collaborations have begun using the \gls{DM} software to run data challenges using precursor data or simulations, and to do performance studies. These activities effectively increase the number of beta-testers of \gls{DM} software products, providing valuable feedback to \gls{DM} on the state of the system.

In order to work effectively with the \gls{DM} software while it is still under development, the science community require:
\begin{itemize}
\item timely access new functionality
\item stable public APIs and schemas in order to build software for User-Generated analyses,
\item patches and bug fixes back-ported to the current stable version of the software,
\item include software provided by external contributors in a software release or distribution (due to the collaborative nature of the project).
\end{itemize}


\subsection{Release Requirements for Data Processing in Operations} \label{sec:procreqs}

The \gls{LDF} will be responsible for generating the necessary data products during commissioning and operations.
The \gls{LDF} requires officially released software to be used in production for the various operational activities.
Software releases will be run in production at National \gls{Center} for Supercomputing Applications (\gls{NCSA}), CC-IN2P3 and in Chile, and possibly at independent Data Access Centers (iDACs).

Release frequencies will depend on the processing type:
\begin{itemize}
\item Prompt Processing requires releases to be available very quick, that could happen on a daily bases, especially in the early days (official releases)
 or a  patch release  for a specific problem may be required during the night (this would require sign off from the \gls{AD} for Science \gls{Operations}).
\item \gls{DRP} processing  must be stable for long periods, currently processing is foreseen to take 9 months.  Before such a long processing run the release must be very well tested and any updates extremely well controlled.
\item Image acquisition and header service, which form part of the image acquisition on the mountain will also need strict change control. Releases for this could be on monthly or even lounge timescales - however if there is a problem a patch will be needed immediately.
\end{itemize}

Patch releases need to be provided with a frequency that depends on the type of processing
and on the urgency of the problems to be fixed.


\subsection{Release Requirements for \gls{LSST} Subsystems in \gls{Operations}.} \label{sec:otherreqs}

Other \gls{LSST} subsystem may be consumers of \gls{DM} software products, for example the Telescope \& Site subsystem software makes use of the \gls{DM} software products.
In order for \gls{DM} to be able to respond correctly to the needs of other \gls{LSST} subsystems, it is important to first identify which \gls{DM} software products are used, and how they are used.


\subsection{Release Requirements for Infrastructural Software} \label{sec:infreqs}

A significant subset of \gls{DM} software products are used to provide services to \gls{LSST} science users and staff but are not directly used to generate \gls{LSST} science data products. An example of this is the software that implements the \gls{LSST} \gls{Science Platform} (\gls{LSP}).

Releases of this type of software are typically on their own cadence and need to be adequately tested before deployment to ensure a stable infrastructure.
The releases may be tied to processing milestones if the services or features thereof are required for the processing (e.g. functionality of the workflow service may be required for \gls{Data Release} processing and features in the \gls{LSP} may be needed for \gls{QA} of data products).

Patch releases may need to be provided depending on the urgency and severity of the problems to be fixed.


\subsection{Release Requirements for non-Operational Project Activities} \label{sec:nonopsreqs}

This includes activities done in preparation for operations, such as commissioning,
large scale integration/validation test campaigns, etc. These activities should use use,  as much as possible,  officially released software.

In some cases however, it is necessary to use non-released software, such as release candidates or stable builds.
In all cases, the software used must be clearly identified (Github revision at least), and the distribution/deployment strictly controlled.


\newpage
\section{Requirements Consolidation} \label{sec:reqdef}

The main purpose of this section is to identify all possible release requirements.
In the first subsection a few general requirements are given.
The second subsection summarizes the requirements given by the stakeholders.

A summary overview of the requirements per stakeholder is given at the end of the section.


\subsection{General Requirements} \label{sec:genreq}

The following general requirements are needed in order to properly implement the release process.


\subsubsection{Software Products Identification Requirement} \label{sec:swid}

All \gls{SW} products shall be clearly and unequivocally identifiable in the source repository (GitHub) and documented.

\citeds{DMTN-106} subsection 2.2 (see \ref{sec:defs}) provides a software product definition that can be used as a starting point to identify the DM software products.
The DM product tree provided together with \citeds{LDM-294}, is available at the following link \url{https://ldm-294.lsst.io/ProductTreeLand.pdf}.

This requirement needs to be fulfilled in order to ensure the applicability of the release policy and process.
If the software products are not properly identified, it will not be possible to do releases.


\subsubsection{Software Release Documentation Requirement} \label{sec:reqdoc}

All software releases shall be properly documented with a software release note.


\subsubsection{Software Release Test Requirement} \label{sec:test}

A software release should be fully tested before making it available for use.
The test should be documented in a test report.


\subsection{Stakeholders Requirements} \label{sec:stakeholdersreqs}

The following list of requirements is derived from the above section \ref{sec:reqs}.


\subsubsection{Releases Schedule Requirement} \label{sec:milestone}

Releases on a software product shall be scheduled in advance.

Two types of release schedule can be identified:

\begin{itemize}
\item Functional Based Release Schedule: a release shall provide a requested functionality.
\item Time-Based Release Schedule: a release shall be provided on a specific date or cadence.
\end{itemize}

In both cases, releases can be tight to project milestones.
Additional releases can be requested to the DM-CCB using the RFC mechanism.


\subsubsection{Patch Release Requirement} \label{sec:backport}

It shall be possible to backport a fix on a stable release and provide a patch release including only the backported fix.


\subsubsection{Third-Party Software Inclusion Requirement} \label{sec:thirdsw}

It shall be possible to include in a software product release or distribution release, a software package provided by a third-party contributor.


\subsubsection{Stable public APIs and Schemas Requirement} \label{sec:stable}

Public APIs and schemas shall be stable and follow a well-defined deprecation mechanism in order to give time to the stakeholder to adapt to the new API.


\subsubsection{License Requirement} \label{sec:license}

DM software shall be released with an appropriate license, that permits to use and redistribution of software provided by external collaborators.


\subsection{Requirements Summary Overview} \label{sec:overview}

The following table gives an overview of the release requirements applicable for each stakeholder.

\setlength\LTleft{-0.4in}
\setlength\LTright{-0.5in}
\begin{longtable}{p{2.4cm}p{1.2cm}p{1.4cm}p{1.4cm}p{1.3cm}p{1.3cm}p{1.3cm}p{1.7cm}p{1.3cm}p{1.3cm}}\hline
&
\textbf{SW Ident.}&\textbf{Release Doc.}&\textbf{Release Test}&\textbf{Funct. Based}&\textbf{Time Based}&\textbf{Patch}&\textbf{3rd Party SW}&\textbf{Stable API} &\textbf{License}\\ \hline
\textbf{LSST Project} \ref{sec:lsstreqs}&
YES                     & YES                         & YES                         &  YES                         &  YES                      & YES             & YES                          &  YES.                   & YES   \\ \hline
\textbf{Science Community} \ref{sec:comreqs}&
YES                     & YES                         &                                 &                                   &  YES                     & YES.            & YES                          &  YES.                   & YES   \\ \hline
\textbf{DM Operations} \ref{sec:procreqs}&
YES                     & YES                         & YES                         & YES                           &                              & YES             & YES                          &  YES(?)                &            \\ \hline
\textbf{Other Operations} \ref{sec:otherreqs}&
YES                     & YES                         & YES                         & YES                           &                              & YES             &                                  &  YES(?)                &            \\ \hline
\textbf{Infrastructure} \ref{sec:infreqs}&
YES                     & YES                         & YES                         &                                   &  YES                     & YES             &                                  &  YES                    &            \\ \hline
\textbf{Non Operations} \ref{sec:nonopsreqs}&
YES                     & YES(?)                     & YES(?)                    & YES                           &  YES                     & YES             & YES                          &  YES                    &            \\ \hline
\hline
\end{longtable}
\setlength\LTleft{0in}
\setlength\LTright{0in}




\newpage
\section{Release Policy} \label{sec:policy}

The following policy is provided taking into account the requirements described in the previous section \ref{sec:reqdef}.


\subsection{Versioning Policy} \label{sec:versinopolicy}

The DM release versioning shall follow Semantic Versioning\footnote{\url{https://semver.org/}} as described in \citeds{DMTN-106}, section 3.3.

This policy partially answers to \ref{sec:stable} requirement.
In general, API stability is a development aspect that needs to be addressed in the \citeds{DevGuideL}.


\subsection{Release Schedule Policy} \label{sec:schedulepolicy}

Major and minor releases should be scheduled accordingly to the requirements of each stakeholder.

A release plan including for each release following information:

\begin{itemize}
\item when a release is expected
\item specify the corresponding milestone or RFC
\item if it is a time-based release: specify the timing when the release shall be done (for example, every 6 months)
\item if it is a functionality based release: specify which functionalities shall be included in the release
\item specify the list of known stakeholders
\end{itemize}

This policy answers requirement \ref{sec:milestone} .


\subsection{Patch Releases and Backporting Policy} \label{sec:patchpolicy}

Patch releases shall be scheduled, tested and approved by the DMCCB as much as possible.
However, the DMPM can authorize urgent changes to the operational systems in case of need.
The backporting capability is a development aspect that needs to be addressed in the \citeds{DevGuideL}.
This policy answers requirement  \ref{sec:backport} .


\subsection{Release Testing Policy} \label{sec:testpolicy}

All releases should be fully tested and test activities documented using Jira ATM.

The scope of the test is to ensure that:

\begin{itemize}
\item the planned functionalities are provided and working
\item there is no regression (reintroduction of previously fixed issues)
\end{itemize}

LSST System Engineering provides in Jira an integrated infrastructure (ATM) that makes it easy to document and execute test activities.

This policy answers requirement \ref{sec:test} .

This policy is not mandatory.
In some cases unit tests may be considered sufficient or, it may not be possible to fully test a software release in isolation.


\subsection{Release Note Policy} \label{sec:notepolicy}

Each release shall be documented by a Release Note.\\
It shall always be possible to collect a minimum set of information to include in the release note and therefore properly identify the release.

\citeds{DMTN-106} subsection 3.2 (see \ref{sec:defs}) provides a definition of software release note that can be used as is or tailed depending on the needs.

This policy answers requirement \ref{sec:reqdoc}.


\subsection{Third-Party Software Policy} \label{sec:thirdpolicy}

In shall be possible to include in a software product or distribution a software package provided by an external collaborator.
The use of all third-party packages shall be approved by the DMCCB using the RFC mechanism.
External software may be included in the following ways:

\begin{itemize}
\item include the package in the environment, if used in the DM code
    \begin{itemize}
    \item this implies that the 3rd party package is available in conda, if not DM should provide a conda recipe.
    \item the software is provided \textit{as is} and DM is not responsible for it
    \end{itemize}
\item include the package in a distribution, as an additional  package \footnote{Currently as EUPS}
    \begin{itemize}
    \item the software is provided \textit{as is} and DM is not responsible for it
    \end{itemize}
\item include the package in a software product, in case it requires specific LSST customization
    \begin{itemize}
    \item DM takes responsibility for the customized package.
    \end{itemize}
\end{itemize}

In all cases a DM person shall be identified as point of contact and reference for the external package.

This policy answers to requirement \ref{sec:thirdsw}.


\subsection{Software License Policy} \label{sec:licensepolicy}

DM shall adopt an appropriate software license in order to fulfill the requirement expressed in section \ref{sec:license}.
Once the license has been identified, this policy can be completed.
\footnote{See discussion in \jira{DM-5031}}


\newpage
\section{Applicability of policies } \label{sec:applicability}

The policies described in the previous section \ref{sec:policy} are mandatory, except where specified as non-mandatory.

The \gls{DM} software products, on which these policies are applicable, are defined in the LDM-294.
The approved product tree is available online at \url{https://ldm-294.lsst.io/ProductTreeLand.pdf}



\subsection{Non Compliance} \label{sec:noncompliance}

Each T/CAM may  specify which policy is not applicable for the specific software product and why.

Each stakeholder shall ensure that the level of compliance provided by the required software products is acceptable for them.

The DM-CCB shall ensure that the available compliance information is agreed between T/CAMs and stakeholders.

Each time a policy is not considered applicable to a software product the effectiveness of the process is reduced and the risks of problems increase.


\newpage
\section{Process} \label{sec:process}


The high-level Release Management plan for \gls{DM} is documented in \citeds{LDM-294}, section 3.6. 
In order to avoid duplications, this process should be maintained only in one document.


\subsection{Release Process Definition}

The release process will be as follows:

\begin{itemize}
\item Major and minor releases will be planned and approved by the DM-CCB.
\begin{itemize}
  \item  DM-CCB will provide  a release plan identifying  milestones requiring software releases.
  \item The DM-CCB will monitor and approve or reject the content of each release. DM-CCB will ensure that  all RFCs that impact  release content are properly escalated.
\end{itemize}
\item Any extra releases, major, minor or patch, has to be requested by the release end users to the DM-CCB via \gls{RFC}. The \gls{RFC} shall contain:
\begin{itemize}
  \item the reason for an extra release
  \item the list of features or fixes which are requested to be implemented in the release
  \item the date the release is requested to be available and specify the urgency.
\end{itemize}
\item The DM-CCB will assess the release request involving the relevant T/CAMs and technical contributor affected.
Urgent release request will be approved in 24 hours, non urgent release requests will be approved in one week's time.
\item The release request may be approved or rejected.
\item In case of approval, the following information will be added in a comment to the RFC:
\begin{itemize}
  \item version identification of the release
  \item date the release will be available (an estimation, that may change)
  \item confirm content of the release
\end{itemize}
\item the DM-CCB will monitor the progress of the release's activity and communicate changes in the release date
\end{itemize}


The DM-CCB meets every week.

Technical aspects and roles (who does what) are addressed int following documents:

\begin{itemize}
\item {\bf regular development}: in the \href{https://developer.lsst.io/}{developer guide}
\item {\bf fix porting}: in the \href{https://developer.lsst.io/}{developer guide}
\item {\bf release note documentation}: in the \href{https://developer.lsst.io/}{developer guide} or a different document
\item {\bf release engineering}: in the DMTN-106 (still draft). \href{https://sqr-016.lsst.io/}{SQR-016} is used for the science pipelines release.
\end{itemize}


\subsection{Urgent Releases}

In the case a very urgent release is required, that can't wait a formal approval of the DM-CCB (24 hours),
a quick decision can be taken by the \gls{DMPM} and a requested fix can be implemented and released immediately, if feasible.

However, an \gls{RFC} has to be filed a posteriori, and the DM-CCB is required to assess it.



\newpage
\appendix

% Include all the relevant bib files.
% https://lsst-texmf.lsst.io/lsstdoc.html#bibliographies
\section{References} \label{sec:bib}
\bibliography{lsst,lsst-dm,refs_ads,refs,books,local}

\newpage
%Make sure lsst-texmf/bin/generateAcronyms.py is in your path
\printglossaries

\end{document}
